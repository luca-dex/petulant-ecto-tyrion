\documentclass[11pt,a4paper]{scrartcl}
\usepackage[utf8]{inputenc}
\usepackage[italian]{babel}

\usepackage{amsmath, amsfonts, amssymb}
\usepackage{graphicx}

\author{L. De Sano, A. Donizetti, M. Scotti}
\title{Risoluzione di sistemi lineari sparsi \\con Python e Scipy}
\date{Maggio 2014}


\begin{document}
\maketitle
\begin{abstract}
Descriviamo l'utilizzo del linguaggio Python e di una sua libreria denominata \emph{scipy} per la risoluzione di sistemi lineari sparsi (ovvero aventi la matrice dei coefficienti associata sparsa).
\end{abstract}

\section*{Python e Scipy}

\subsection*{Python}
% perchè python
% caratteristiche e storia del linguaggio python
% utilizzo in ambito computazione scientifica
% pro \ contro

\subsection*{Scipy}
% descrizione generale
% cosa contiene e cosa fa
% come si installa
% come è documentato
% il progetto è attivo [dati commits / mailing lists]


\section*{Risoluzione di sistemi sparsi}

\subsection*{Lettura dei file e memorizzazione}
% performance parser hand-made vs mmread integrato
% problema conversione formato MatrixMarket
% script di conversione con tempi di esecuzione

\subsection*{Metodi diretti}
% metodi disponibile
% tempi di esecuzione
% errori sulla soluzione

\subsection*{Metodi iterativi}
% metodi disponibili
% tempi di esecuzione
% errori nella soluzione

\subsection*{Problematiche incontrate}
% matrice quasi singolare 

\subsection*{Conclusione}
% conclusione

\end{document}